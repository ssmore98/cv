\documentclass[10pt]{article}
\usepackage[colorlinks=false]{hyperref}
\usepackage{pifont}
\usepackage{longtable}
\usepackage[width=7in,height=9in]{geometry}
\pagestyle{empty}
\begin{document}
\begin{center}
{\bf \Large SACHIN MORE} \\
141 Howard Street, Northborough, MA 01532-1316 \\
{\ding{38} \em +1-508-335-6351} \\
{\em sachinsureshmore@gmail.com}
\end{center}
\rule{\textwidth}{1pt}
%\begin{center}
%{\bf SUMMARY}
%\end{center}

\begin{longtable}{p{91pt}p{390pt}}
\\\multicolumn{2}{p{493pt}}{
    \begin{itemize}
        \item 20+ years of experience as software developer
        \item Hands-on problem solving skills in difference areas
        \item Strong algorithms and data structures background specializing
            in performance optimization
        \item A strong record of innovation with more than twenty patents
    \end{itemize}
}\\
\\\\
{\bf Skills}           & C/C++, Python, Linux, GitLab, Jira,
RTOS, Simulation Software, Performance Analysis, Optimization Algorithms,
System Software, Cross-functional Communication\\\\
{\bf Accomplishments}  & {\bf Storage System Firmware}\\
                       & {New ideas were proposed, implemented and evaluated to improve various parts of 
                          the EMC Symmetrix storage array microcode. Some ideas made
                          fundamental changes to how the product works:
                          \begin{itemize}
                              \item The underlying data structures were changed to
                                  allow the code to quickly locate dirty data in cache that
                                  was no longer useful. (U.S. patents 6,865,648 and 7,437,515)
                              \item A new alogorithm was designed to decide which copy of the data
                                  to read when multiple copies are present to distribute the
                                  read activity evenly across the resources. (U.S. patent 6,954,833)
                              \item A new cache management system was implemented to replace the existing
                                  multi-LRU cache to improve parallel access to the shared memory cache. (U.S. patent 7,177,853)
                          \end{itemize}
                          Others were incremental changes:
                          \begin{itemize}
                              \item The multi-LRU caching mechanism was modified to strike a better balance
                                  between the performance (retaining frequently accessed data) and the cost
                                  (memory accesses needed to retain data in the cache). (U.S. patent 6,715,039)
                              \item The existing data pre-fetching algorithm was enhanced for the cases where
                                  workloads contain short bursts of contigous data accesses. (U.S. patent 6,721,870)
                              \item The RAID data rebuild algorithm was modified to improve 
                                  its performance when the storage system is under heavy load (U.S. patent 7,406,574)
                          \end{itemize}
                          The development was done in C under a proprietary
                          BSD-derived RTOS and toolchain. The algorithms and data structures used were chosen
                          based on both space and time efficiency.}\\
                          \\
                          \newpage
                       & {\bf Simulation Software}\\
                       & {
                          \begin{itemize}
                              \item {\em Cache Simulator} This software simulated the behaviour of the 
                                  shared memory cache model used by the EMC Symmetrix/VMAX storage system.
                              \item {\em Hard Disk Drive Simulator} This software simulated the different
                                  hard disk drives used by the EMC Symmetrix/VMAX storage system.
                              \item {\em Metadata Paging Simulator} This software simulated the metadata paging
                                  mechanism used by the EMC Symmetrix/VMAX storage system.
                          \end{itemize}
                          These applications were used to validate proposed optimization techniques for the
                          EMC Symmetrix/VMAX storage system. They were written in C++ for Fedora Linux.}\\
                     \\
                       & {\bf Software Appliances}\\
                       & {These appliances were based on CentOS Linux and were written using C++/GTK.
                     \begin{itemize}
                         \item The {\em Secure Erase Appliance for SSDs} was
                             part of the launch of SSDs in the EMC storage products lineup.
                             The appliance is used by the {\em EMC certified secure erase service} to
                             decommission/re-purpose SSDs in the field. The design emphasis was on
                             data security and the appliance was validated and certified by
                             an independent third party authority.
                         \item The {\em Disk Cloning Appliance} was used for bulk
                             replacement of faulty disks in a storage array. The appliance allows the
                             storage array to function normally while the drives are being replaced.
                             The design emphasis was on speed and accuracy. 150 appliance instances
                             were used to replace several hundred thousand
                             drives in storage arrays across the globe in little over a year.
                 \end{itemize}}\\
                       & {\bf Hardware Component Qualification Software}\\
                       & {The hardware component qualification process used Python scripts running under Centos Linux.
                          The scripts tested:
                          \begin{itemize}
                              \item Hard disk drives
                              \item Solid state drives
                              \item Fibre channel and SAS HBAs
                              \item Ethernet NICs
                              \item Infiniband HBAs.
                          \end{itemize}

                          This software also included an I/O driver for doing long-term testing of disk drives. It was written in C
                          and ran on Centos Linux.
                              }\\
                      \\
\end{longtable}
\newpage
\section{Work Experience}

\cventry{2/2016 - 3/2020}{Consultant Technologist}{Infinidat Inc.}{Waltham, MA}{Platform Research and Development}
{
{\bfseries Responsibilities} Participate in the creation and updates of qualification and test procedures and content,
and/or lead them
for HDDs, SSDs, PCIe cards and Servers. Create low-level tools and tests for manufacturing and qualifications.
Handle specific issues with the vendors like firmware bugs and suspected design issues.
Monitor the industry for new technologies. Participate in vendor technology updates and design reviews.
Participate in recurring calls with selected vendors.
Participate in the internal design reviews and code reviews.}
\cventry{2/2015 - 1/2016}{Consultant Technologist}{EMC Corporation}{Hopkinton, MA}{VMAX Systems Engineering}
{
}
\cventry{10/2014 - 2/2015}{Consultant Technologist}{EMC Corporation}{Hopkinton, MA}{Office of CTO, Core Technologies Division}
{
}
\cventry{4/2013 - 10/2014}{Consultant Technologist}{EMC Corporation}{Hopkinton, MA}{Office of Strategy and Technology, Enterprise and Mid-range Storage Division}
{{\bfseries Responsibilities}
Represent own specific area of expertise as well as EMC's broader technical and business vision in analyzing technology
trends over a three-to-five year time span for influence and impact on EMC products, strategy, and technology.
}
\cventry{3/2012 - 4/2013}{Consulting Performance Engineer}{EMC Corporation}{Hopkinton, MA}{Innovation and Systems Engineering,
Enterprise Storage Division}
{
}
\cventry{12/2006 - 3/2012}{Principal Performance Engineer}{EMC Corporation}{Hopkinton, MA}{Innovation and Systems
Engineering, Enterprise Storage Division}
{
}
\cventry{12/2003 - 12/2006}{Principal Performance Engineer}{EMC Corporation}{Hopkinton, MA}{Performance Research Group,
Symmetrix Business Unit}
{
}
\cventry{8/2000 - 12/2003}{Senior Performance Engineer}{EMC Corporation}{Hopkinton, MA}{Performance Research Group,
Symmetrix Business Unit}
{{\bfseries Responsibilities}
\begin{itemize}
\item  Design/implement/maintain simulation software for various parts of EMC's storage system product
\item  Design/simulate/implement/analyze performance improvements for EMC's storage system product
\end{itemize}
}
\cventry{9/1996 - 6/2000}{Research Assistant}{Department of Electrical and Computer Engineering}
{Northwestern University}{Evanston, IL}
{\begin{itemize}
	\item System administrator for an IBM SP2 and SGI Origin belonging to the 
		Center for Parallel and Distributed Computing at Northwestern University.
	\item Developed methods for efficient access to large objects in an object-oriented
		database running on a distributed memory parallel computer.
	\end{itemize}}
\cventry{10/1994 - 8/1996}{Research Assistant}{Department of Electrical and
Computer Engineering}{Syracuse University}{Syracuse, NY}{
Designed and evaluated a multi-threaded runtime library for parallel I/O.
It separated the compute and I/O tasks in two separate threads of control.
Multi-threading in the design permits:
\begin{itemize}
	\item asynchronousI/O even if the
underlying file system does not support asynchronous I/O
\item  copy avoidance from the I/O thread to the compute thread by sharing address
space 
\item a capability to perform collective I/O asynchronously without 
blocking the compute threads.
\end{itemize}
The library also implements collective I/O which maximize load balance and
concurrency while reducing communication overhead in an integrated  fashion.
The library was evaluated on IBM SP2 for various data distributions and
access patterns. The results show that there is a tradeoff between the amount of
concurrency in I/O and the buffer size designated for I/O; and there is an
optimal buffer size beyond which benefits of larger requests diminish due to
large communication overhead.}

\section{Education}   
\cventry{1996--2000}{Doctor of Philosophy}{Northwestern University}{Evanston, IL}{}{Computer Engineering}
\cventry{1994--1996}{Master of Science}{Syracuse University}{Syracuse, NY}{}{Computer Engineering}
\cventry{1987--1991}{Bachelor of Engineering}{Savitribai Phule Pune University}{Pune, India}{}{Computer Engineering}
\section{Awards and Recognition}
\cventry{2008}{President's Award at EMC Corp.}{For innovations related to solid state drive technology.}{}{}{}


%% \begin{longtable}{p{ 90pt}p{299pt}r}
%% {\bf Work Experience} & {\bf Consultant Technologist} & {2/2016 - 3/2020}\\
%%                       & \multicolumn{2}{p{4.5in}}{\em Platform Research and Development} \\
%%                       & {\em Infinidat Inc., Waltham, MA} \\
%% \\
%%                       & {\bf Consultant Technologist} & {2/2015 - 1/2016}\\
%%                       & \multicolumn{2}{p{4.5in}}{\em VMAX Systems Engineering} \\
%%                       & {\em EMC Corporation, Hopkinton, MA} \\
%% \\
%%                       & {\bf Consultant Technologist} & {10/2014 - 2/2015}\\
%%                       & \multicolumn{2}{p{4.5in}}{\em Office of CTO, Core Technologies Division} \\
%%                       & {\em EMC Corporation, Hopkinton, MA} \\
%% \\
%%                       & {\bf Consultant Technologist} & {4/2013 - 10/2014}\\
%%                       & \multicolumn{2}{p{4.5in}}{\em Office of Strategy and Technology, Enterprise and Mid-range Storage Division} \\
%%                       & {\em EMC Corporation, Hopkinton, MA} \\
%% \\
%%                       & {\bf Consulting Performance Engineer} & {3/2012 - 4/2013}\\
%%                       & \multicolumn{2}{p{4.5in}}{\em Innovation and Systems Engineering, Enterprise Storage Division} \\
%%                       & {\em EMC Corporation, Hopkinton, MA} \\
%% \\
%%                       & {\bf Principal Performance Engineer} & {12/2006 - 3/2012}\\
%%                       & \multicolumn{2}{p{4.5in}}{\em Innovation and Systems Engineering, Enterprise Storage Division} \\
%%                       & {\em EMC Corporation, Hopkinton, MA} \\
%% \\
%%                       & {\bf Principal Performance Engineer} & {12/2003 - 12/2006}\\
%%                       & \multicolumn{2}{p{4.5in}}{\em Performance Research Group, Symmetrix Business Unit} \\
%%                       & {\em EMC Corporation, Hopkinton, MA} \\
%% \\
%%                       & {\bf Senior Performance Engineer} & {8/2000 - 12/2003}\\
%%                       & \multicolumn{2}{p{4.5in}}{\em Performance Research Group, Symmetrix Business Unit} \\
%%                       & {\em EMC Corporation, Hopkinton, MA} \\
%% \\
%%                       & {\bf Intern} & {Summer 1998}\\
%%                       & {\em Bell Laboratories, Murray Hill, NJ} \\
%% \\
%% %                      & {\bf Software Engineer} & {2/1994 - 7/1994}\\
%% %                      & {\em Datamatics Ltd., Mumbai, India} \\
%% %\\
%% %                      & {\bf System Executive} & {2/1993 - 1/1994}\\
%% %                      & {\em Spectrum Business Support Pvt. Ltd., Mumbai, India} \\
%% %\\
%% %                      & {\bf Freelance Work} & {7/1991 - 1/1993}\\
%% %                      & {\em Shree Software Consultants, Pune, India} \\
%% %\\
%% {\bf Education}       & {\bf Doctor of Philosophy} & 2000\\
%%                       & {\em Computer Engineering} \\
%%                       & {Northwestern University, Evanston, IL} \\
%% \\
%%                       & {\bf Master of Science} & 1996\\
%%                       & {\em Computer Engineering} \\
%% 	              & {Syracuse University, Syracuse, NY} \\
%% \\
%%                       & {\bf Bachelor of Engineering} & 1991\\
%%                       & {\em Computer Engineering} \\
%% 	              & {University of Poona, India} \\
%% \\
%% {\bf Awards}         & {\bf President's Award at EMC Corp.} & 2008 \\
%%                      & \multicolumn{2}{p{3.35in}}{For innovations related to solid state drive technology.}\\
%% \\
%% \end{longtable}

\end{document}
