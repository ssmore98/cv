\documentclass[10pt,a4paper,sans]{moderncv}
\moderncvstyle{classic}
\moderncvcolor{black}
\nopagenumbers{}
\usepackage[scale=0.75]{geometry}
\name{Sachin}{More}
\address{141 Howard Street}{Northborough MA 01532}{USA}
\phone[mobile]{+1~(508)~335~6351}
\phone[fixed]{+1~(508)~393~9493}
\email{sachinsureshmore@gmail.com}
\social[linkedin]{samore}
%\social[github]{ssmore98}



\quote{
    \begin{itemize}
        \item 20+ years of experience as software developer
        \item Hands-on problem solving skills in difference areas
        \item Strong algorithms and data structures background specializing
            in performance optimization
        \item A strong record of innovation with more than twenty patents
    \end{itemize}
}
\begin{document}
\makecvtitle
\section{Skills} 
\cvitem{Programming Languages}{C/C++, Python}
\cvitem{Operating Systems}{Linux, RTOS}
\cvitem{Development Tools}{GitLab, Jira, make}
\cvitem{}{Simulation Software, Performance Analysis, Optimization Algorithms,
System Software, Cross-functional Communication}

\section{Accomplishments} 
\cvitem{Storage System Firmware}{New ideas were proposed, implemented and evaluated to improve various parts of 
                          the EMC Symmetrix storage array microcode. Some ideas made
                          fundamental changes to how the product works:
                          \begin{itemize}
                              \item The underlying data structures were changed to
                                  allow the code to quickly locate dirty data in cache that
                                  was no longer useful. (U.S. patents 6,865,648 and 7,437,515)
                              \item A new alogorithm was designed to decide which copy of the data
                                  to read when multiple copies are present to distribute the
                                  read activity evenly across the resources. (U.S. patent 6,954,833)
                              \item A new cache management system was implemented to replace the existing
                                  multi-LRU cache to improve parallel access to the shared memory cache. (U.S. patent 7,177,853)
                          \end{itemize}
                          Others were incremental changes:
                          \begin{itemize}
                              \item The multi-LRU caching mechanism was modified to strike a better balance
                                  between the performance (retaining frequently accessed data) and the cost
                                  (memory accesses needed to retain data in the cache). (U.S. patent 6,715,039)
                              \item The existing data pre-fetching algorithm was enhanced for the cases where
                                  workloads contain short bursts of contigous data accesses. (U.S. patent 6,721,870)
                              \item The RAID data rebuild algorithm was modified to improve 
                                  its performance when the storage system is under heavy load (U.S. patent 7,406,574)
                          \end{itemize}
                          The development was done in C under a proprietary
                          BSD-derived RTOS and toolchain. The algorithms and data structures used were chosen
                          based on both space and time efficiency.}
\cvitem{Simulation Software}{
                          \begin{itemize}
                              \item {\em Cache Simulator} This software simulated the behaviour of the 
                                  shared memory cache model used by the EMC Symmetrix/VMAX storage system.
                              \item {\em Hard Disk Drive Simulator} This software simulated the different
                                  hard disk drives used by the EMC Symmetrix/VMAX storage system.
                              \item {\em Metadata Paging Simulator} This software simulated the metadata paging
                                  mechanism used by the EMC Symmetrix/VMAX storage system.
                          \end{itemize}
                          These applications were used to validate proposed optimization techniques for the
                          EMC Symmetrix/VMAX storage system. They were written in C++ for Fedora Linux.}
\cvitem{Software Appliances}
                       {These appliances were based on CentOS Linux and were written using C++/GTK.
                     \begin{itemize}
                         \item The {\em Secure Erase Appliance for SSDs} was
                             part of the launch of SSDs in the EMC storage products lineup.
                             The appliance is used by the {\em EMC certified secure erase service} to
                             decommission/re-purpose SSDs in the field. The design emphasis was on
                             data security and the appliance was validated and certified by
                             an independent third party authority.
                         \item The {\em Disk Cloning Appliance} was used for bulk
                             replacement of faulty disks in a storage array. The appliance allows the
                             storage array to function normally while the drives are being replaced.
                             The design emphasis was on speed and accuracy. 150 appliance instances
                             were used to replace several hundred thousand
                             drives in storage arrays across the globe in little over a year.
                 \end{itemize}}
\cvitem{Hardware Component Qualification Software}
{The hardware component qualification process used Python scripts running under Centos Linux.
                          The scripts tested:
                          \begin{itemize}
                              \item Hard disk drives
                              \item Solid state drives
                              \item Fibre channel and SAS HBAs
                              \item Ethernet NICs
                              \item Infiniband HBAs.
                          \end{itemize}
                          This software also included an I/O driver for doing long-term testing of disk drives. It was written in C
                          and ran on Centos Linux.}
\end{document}
\end{document}
