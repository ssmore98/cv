\documentclass[10pt,a4paper,sans]{moderncv}
\moderncvstyle{classic}
\moderncvcolor{black}
\nopagenumbers{}
\usepackage[scale=0.79]{geometry}
\name{Sachin}{More}
\address{141 Howard Street}{Northborough MA 01532}{USA}
\phone[mobile]{+1~(508)~335~6351}
\phone[fixed]{+1~(508)~393~9493}
\email{sachinsureshmore@gmail.com}
\social[linkedin]{samore}
\social[github]{ssmore98}



\quote{
}

\newcommand{\emc}{\href{https://en.wikipedia.org/wiki/Dell_EMC}{EMC Corporation}}
\newcommand{\hopkinton}{\href{https://www.hopkintonma.gov/}{Hopkinton, MA}}
\newcommand{\evanston}{\href{https://www.cityofevanston.org/}{Evanston, IL}}
\newcommand{\syracuse}{\href{http://www.syrgov.net}{Syracuse, NY}}
\newcommand{\northwestern}{\href{https://www.northwestern.edu/}{Northwestern University}}
\newcommand{\suniv}{\href{https://www.syracuse.edu/}{Syracuse University}}

\begin{document}
\makecvtitle
\section{Skills}
\cvitem{Programming Languages}{C/C++, Python, MATLAB}
\cvitem{Operating Systems}{Linux}
\cvitem{Development Tools}{gcc/g++, git, make}
\cvitem{Interests}{Algorithms, Data Structures, Data Analytics, Deep Learning,
	Simulation, Performance Analysis, Parallel and Distributed Computing,
    Multi-threading}

%\section{Certifications}
%\cvitem{IBM}{\href{https://courses.edx.org/certificates/6d0b889832c94649928465bef5342974}
%{DV0101EN: Visualizing Data with Python}}
%\cvitem{}{\href{https://courses.edx.org/certificates/c9d9d9ee599040b5836fe477100e2245}
%{DA0101EN: Analyzing Data with Python}}
%\cvitem{}{\href{https://courses.edx.org/certificates/e4ab2175ff03440899b2444ee6098ec1}
%{PY0101EN: Python Basics for Data Science}}
%\cvitem{}{\href{https://courses.edx.org/certificates/9160fffad96d4ce793e08ed62364f56e}
%{DS0101EN: Introduction to Data Science}}

\section{Work Experience}

\cventry{1/2021 - Current}{Senior Staff Product Manager}{\href{https://www.kioxia.com/en-us/top.html}{KIOXIA America Inc.}}
{\href{https://www.sanjoseca.gov/}{San Jose, CA}} {\href{https://kumoscale.kioxia.com/en/}{KumoScale Storage System}}
{
	\begin{itemize}
		\item Technical evangelist for the product.
		\item Technical advisor for the solutions engineering team.
		\item Represent product management in customer meetings.
		\item Shared responsibility for requirements documentation for the R\&D team.
	\end{itemize}
}

\cventry{2/2016 - 3/2020}{Consultant Software Technologist}{\href{http://www.infinidat.com}{Infinidat Inc.}}
{\href{https://www.city.waltham.ma.us/}{Waltham, MA}} {Platform Research and Development}
{
	\begin{itemize}
		\item Defined and implemented hardware qualification processes for
			hardware components. The development was done in
			\texttt{C/C++/Python} on a \texttt{CentOS Linux} platform. The code
			was maintained under \texttt{\href{https://about.gitlab.com/}{GitLab}}.
		\item Implemented monitoring services for customer systems. The software
			was written in \texttt{Python} and the data collection was done
			in \texttt{JSON} over a \texttt{REST} API. The collected data was
			analyzed using
			\texttt{\href{https://pandas.pydata.org/}{pandas}},
			\texttt{\href{https://numpy.org/}{numpy}} and
			\texttt{\href{https://matplotlib.org/}{matplotlib}} and forecasts were
			provided to the upper management.
	\end{itemize}
}
\cventry{4/2013 - 1/2016}{Consultant Technologist}{\emc}{\hopkinton}{}
{
	\begin{itemize}
		\item \href{https://patentimages.storage.googleapis.com/5d/7d/f0/1dbfff1e0caad8/US8375187.pdf}{Invented
			an algorithm} for scheduling I/Os to SSDs to improve
			their performance in a RAID configuration.
		\item \href{https://patentimages.storage.googleapis.com/aa/07/25/9604f6bd305e36/US8838887.pdf}{Designed
			a new data layout scheme for hard disk drives} to improve
			their performance in a tiered storage system.
			The scheme was implemented and tested using an I/O driver written
			in \texttt{C/C++} on \texttt{CentOS Linux} platform using the \texttt{sg} device driver.
	\end{itemize}
}
\cventry{3/2012 - 4/2013}{Consulting Performance Engineer}{EMC Corporation}{Hopkinton, MA}
{Innovation and Systems Engineering, Enterprise Storage Division}
{
	\begin{itemize}
        \item Proposed and implemented the {\em Disk Cloning Appliance} used for bulk
            replacement of faulty disks in a storage array. The appliance was
			developed in \texttt{C++} under \texttt{CentOS Linux} and used \texttt{GTK+} for the user
			interface.
		\item Implemented a simulator to evaluate the proposed metadata paging
			scheme in EMC VMAX Storage System. The software was written in
			\texttt{C++} on a \texttt{CentOS Linux} platform.
	\end{itemize}
}
\cventry{12/2003 - 3/2012}{Principal Performance Engineer}{\emc}{\hopkinton}{Innovation and Systems
Engineering, Enterprise Storage Division}
{
	\begin{itemize}
		\item Designed and implemented the {\em Secure Erase Appliance for SSDs}
			for the {\em EMC certified secure erase service} as part of the
			launch of SSDs in the EMC storage products lineup. The appliance was
			developed in \texttt{C++} under \texttt{CentOS Linux} and used
			\texttt{GTK+} for the user interface.
		\item Proposed, prototyped and evaluated
			\href{https://patentimages.storage.googleapis.com/35/28/54/3ec9177625fc05/US7406574.pdf}
			{modifications to the RAID data rebuild algorithm}
			in the EMC Symmetrix Storage System to improve its performance when
			the storage system is under heavy load. The
			development was done in \texttt{C} under a proprietary BSD-derived RTOS.
		\item Maintained Cache Simulator software that simulated the behavior of the
			shared memory cache model used by the EMC Symmetrix/VMAX storage system.
			The software was originally written in \texttt{C}. Ported it to \texttt{C++}.
		\item Developed and maintained Hard Disk Simulator software that
			simulated the behavior of different hard disk drives used by the
			EMC Symmetrix/VMAX storage system. The software was written in \texttt{C++}.
	\end{itemize}
}
\cventry{8/2000 - 12/2003}{Senior Performance Engineer}{\emc}{\hopkinton}{Performance Research Group,
Symmetrix Business Unit}
{
	\begin{itemize}
		\item Proposed, prototyped and evaluated
			\href{https://patentimages.storage.googleapis.com/59/4b/0d/64a4ee1769c704/US7437515.pdf}
			{changes to how the EMC Symmetrix Storage System manages dirty data}.
			The development was done in \texttt{C} under a proprietary BSD-derived RTOS.
		\item Proposed, prototyped and evaluated
			\href{https://patentimages.storage.googleapis.com/e6/95/d6/fda2d287286a2d/US6954833.pdf}
			{a new algorithm in the EMC Symmetrix Storage System schedules read
			operations on disk drives in \texttt{RAID-1} configuration}.
			The development was done in \texttt{C} under a proprietary BSD-derived RTOS.
		\item Proposed, prototyped and evaluated 
			\href{https://patentimages.storage.googleapis.com/c6/ad/4f/ad550b2e923d0d/US6721870.pdf}
			{enhancement to the existing data pre-fetching algorithm
			in the EMC Symmetrix Storage System} for the cases where workloads contain short bursts of
			contiguous data accesses. The development was done in \texttt{C} under a proprietary BSD-derived RTOS.
		\item Prototyped and evaluated
			\href{https://patentimages.storage.googleapis.com/2a/d0/49/84736ddbd4d47b/US6715039.pdf}
			{changes to the multi-LRU caching mechanism in the EMC Symmetrix Storage
			System} to strike a better balance between the performance (retaining
			frequently accessed data) and the cost
			(memory accesses needed to retain data in the cache).
			The development was done in \texttt{C} under a proprietary BSD-derived RTOS.
	\end{itemize}
}
\cventry{9/1996 - 6/2000}{Research Assistant}{\href{https://www.mccormick.northwestern.edu/electrical-computer/academics/graduate/}{Department of Electrical and Computer Engineering}}
{\northwestern}
{\evanston}
{\begin{itemize}
		\item System administrator for an \href{https://en.wikipedia.org/wiki/IBM_Scalable_POWERparallel}
			{IBM SP2} and \href{https://en.wikipedia.org/wiki/SGI_Origin_2000}
			{SGI Origin} belonging to the
		Center for Parallel and Distributed Computing at Northwestern University.
	\item Researched methods for efficient access to large objects in an object-oriented
		database running on a distributed memory parallel computer. The
		software was written in \texttt{C++}.
	\end{itemize}}
	\cventry{10/1994 - 8/1996}{Research Assistant}{\href{https://eng-cs.syr.edu/}
	{Department of Electrical and Computer Engineering}}{\suniv}{\syracuse}
{Developed and evaluated a multi-threaded run-time library for parallel I/O that
separated the compute and I/O tasks in two separate threads of control. The
software was written in \texttt{C}.}
\cventry{2/1994 - 7/1994}{Software Engineer}
{\href{https://www.datamatics.com/}{Datamatics Ltd.}}
{\href{https://en.wikipedia.org/wiki/Mumbai}{Mumbai, India}}{}
{Worked on a document sharing system over WAN. The software was written in \texttt{C}.}
\cventry{2/1993 - 1/1994}{System Executive}{Spectrum Business Support Pvt. Ltd.}
{\href{https://en.wikipedia.org/wiki/Mumbai}{Mumbai, India}}
{}{Was member of a team that implemented a archive retrieval system.
Wrote the communication layer of the software that connected the server running
on a SUN system to the client PC over a telephone line. The software was
written in \texttt{C}.}
\section{Education}
\cventry{1996--2000}{Doctor of Philosophy}{\northwestern}{\evanston}{}{Computer Engineering}
\cventry{1994--1996}{Master of Science}{\suniv}{\syracuse}{}{Computer Engineering}
\cventry{1987--1991}{Bachelor of Engineering}{\href{http://www.unipune.ac.in/}
{Savitribai Phule Pune University}}
{\href{https://en.wikipedia.org/wiki/Pune}{Pune, India}}{}{Computer Engineering}
\end{document}
