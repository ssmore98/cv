\documentclass[12pt,a4paper,sans]{moderncv}
\moderncvstyle{classic}
\moderncvcolor{black}
\nopagenumbers{}
\usepackage[scale=0.75]{geometry}
\name{Sachin}{More}
\address{141 Howard Street}{Northborough MA 01532}{USA}
\phone[mobile]{+1~(508)~335~6351}
\phone[fixed]{+1~(508)~393~9493}
\email{sachinsureshmore@gmail.com}
\social[linkedin]{samore}
\social[github]{ssmore98}



\title{Python Experience}
\begin{document}
\makecvtitle
\section{Overview}
\cvitem{}{
My work at Infinidat Inc. involved (among other things) designing and 
implementing hardware qualification processes for the off-the-shelf hardware
components used to manufacture the flagship product. The hardware components
included:
\begin{itemize}
\item x86-based Servers
\item Fibre channel HBAs
\item SAS HBAs
\item Ethernet adapters (Gigabit and 10Gbps)
\item Hard disk drives (SAS)
\item Solid-state drives (SAS/SATA)
\item Infiniband HBAs
\end{itemize}
}
\section{Development Environment}
\cvitem{}{
The development and production platform was CentOS Linux 7. The development was done 
in Python 2/3 (couple of low-level drivers were written C/C++). I was responsible
for the development, maintenance and support of the software and involved in
the qualification logistics for four years.}

\section{Features}
\cvitem{}{
The software ran on Ubuntu Linux and used parallel and remote execution to test the
components on the production system. Some important modules used included:}
\cventry
{}{threading}{}{}{}{For parallel execution of local compute-intensive tasks.}
\cventry
{}{subprocess}{}{}{}{For remote, parallel task execution over ssh and ipmi.}
\cventry
{}{httplib2}{}{}{}{For accessing REST APIs}
\cventry
{}{matplotlib, numpy, docx}{}{}{}{To prepare charts and reports.}
\cventry
{}{re}{}{}{}{For scraping command output.}
\cventry
{}{pickle, json, xml, tarfile, gzip}{}{}{}{To store results, configuration, program state etc.}
\cventry
{}{cron}{}{}{}{To keep track of log running tests}

\end{document}


