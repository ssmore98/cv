\documentclass[10pt]{article}
\usepackage[colorlinks=false]{hyperref}
\usepackage{pifont}
\usepackage{longtable}
\usepackage[width=7in,height=9in]{geometry}
\pagestyle{empty}
\begin{document}
\begin{center}
{\bf \Large SACHIN MORE} \\
141 Howard Street, Northborough, MA 01532-1316 \\
{\ding{38} \em +1-508-335-6351} \\
{\em sachinsureshmore@gmail.com}
\end{center}
\rule{\textwidth}{1pt}
%\begin{center}
%{\bf SUMMARY}
%\end{center}


\begin{longtable}{p{ 90pt}p{299pt}r}
{\bf Patents}       & {\bf U.S. Patent 6,591,287} & 7/2003 \\
                    & \multicolumn{2}{p{4.5in}}{\em Method to increase the efficiency of job sequencing from sequential storage}\\
                    & \multicolumn{2}{p{4.5in}}{This patent describes a technique to efficiently accessing large scientific data sets
						 stored in a magnetic tape library.} \\
\\
                    & {\bf U.S. Patent 6,715,039} & 3/2004 \\
                    & \multicolumn{2}{p{4.5in}}{\em Cache slot promotion in a replacement queue cache using determinations of probabilities and costs}\\
                    & \multicolumn{2}{p{4.5in}}{This patent describes a technique to manage data stored in an Integrated Cache Disk Array (ICDA). It 
						is used to determine how soon a piece of data will be accessed again.}\\
\\
                    & {\bf U.S. Patent 6,721,870} & 4/2004 \\
                    & \multicolumn{2}{p{4.5in}}{\em Prefetch algorithm for short sequences}\\
                    & \multicolumn{2}{p{4.5in}}{This patent describes a technique to predict future data accesses for a logical volume when the accesses
						 are small, few and concentrated in a small area. It is used by the microcode in the EMC Symmetrix product
						line.}\\
\\
                    & {\bf U.S. Patent 6,769,054} & 7/2004 \\
                    & {\bf U.S. Patent 7,213,113} & 5/2007 \\
                    & \multicolumn{2}{p{4.5in}}{\em System and method for preparation of workload data for replaying in a data storage environment}\\
                    & \multicolumn{2}{p{4.5in}}{These patents describe a technique to help replicate a customer problem in a lab environment. The
						method takes a trace of the I/O activity in the customer environment and transforms it into
						an I/O activity trace appropriate for the lab environment.}\\
\\
                    & {\bf U.S. Patent 6,865,648} & 3/2005 \\
                    & {\bf U.S. Patent 7,437,515} & 10/2008 \\
                    & \multicolumn{2}{p{4.5in}}{\em Data structure for write pending} \\
                    & \multicolumn{2}{p{4.5in}}{These patents describe an efficient way to manage dirty data in an Integrated Cache Disk Array (ICDA).}\\
\\
                    & {\bf U.S. Patent 6,954,833} & 10/2005 \\
                    & \multicolumn{2}{p{4.5in}}{\em Expedited dynamic mirror service policy} \\
                    & \multicolumn{2}{p{4.5in}}{This patent describes a complete overhaul of the {\em Dynamic Mirror Service Policy} (DMSP) in the EMC Symmetrix
						product line. DMSP is used for orchestrating reading data from RAID-1 devices.}\\
\\
                    & {\bf U.S. Patent 7,177,853} & 2/2007 \\
                    & \multicolumn{2}{p{4.5in}}{\em Cache management via statistically adjusted time stamp queue} \\
                    & \multicolumn{2}{p{4.5in}}{This patent forms the basis for the cache management scheme used in EMC Symmetrix product line.}\\
\\
\newpage
                    & {\bf U.S. Patent 7,281,086} & 10/2007 \\
                    & {\bf U.S. Patent 7,293,136} & 11/2007 \\
                    & \multicolumn{2}{p{4.5in}}{\em Management of two-queue request structure for quality of service in disk storage systems} \\
                    & {\bf U.S. Patent 8,935,490} & 1/2015 \\
                    & \multicolumn{2}{p{4.5in}}{\em Disk access quality of service} \\
                    & \multicolumn{2}{p{4.5in}}{These patents describe a method to implement {\em Quality of Service} (QOS) for the disks used by
						EMC Symmetrix product line.}\\
\\
                    & {\bf U.S. Patent 7,406,574} & 7/2008 \\
                    & \multicolumn{2}{p{4.5in}}{\em Management of invalid tracks} \\
                    & \multicolumn{2}{p{4.5in}}{This patent describes a method to find inconsistent pieces of data in a RAID protected system like
						EMC Symmetrix.}\\
\\
                    & {\bf U.S. Patent 7,552,280} & 6/2009 \\
                    & \multicolumn{2}{p{4.5in}}{\em Asymmetrically interleaving access to redundant storage devices} \\
                    & \multicolumn{2}{p{4.5in}}{This patent describes an enhancement of the {\em Dynamic Mirror Service Policy} scheme used
						by EMC Symmetrix to read data from RAID-1 protected device.}\\
\\
                    & {\bf U.S. Patent 7,640,342} & 12/2009 \\
                    & \multicolumn{2}{p{4.5in}}{\em System and method for determining configuration of one or more data storage systems} \\
                    & \multicolumn{2}{p{4.5in}}{The patent describes a method to configure a data storage system based on observed performance 
						metrics. It is employed  in a software used by field personnel for configuring EMC Symmetrix 
						data storage system.} \\
\\
                    & {\bf U.S. Patent 7,882,373} & 2/2011 \\
                    & \multicolumn{2}{p{4.5in}}{\em System and method of reducing power consumption in a storage system through shortening of seek distances} \\
                    & {\bf U.S. Patent 8,060,759} & 11/2011 \\
                    & \multicolumn{2}{p{4.5in}}{\em System and method of managing and optimizing power consumption in a storage system} \\
                    & {\bf U.S. Patent 9,158,466} & 10/2015 \\
                    & \multicolumn{2}{p{4.5in}}{\em Power-saving mechanisms for a dynamic mirror service policy} \\
                    & \multicolumn{2}{p{4.5in}}{These patents describe ways to reduce power consumption by hard disk drives in a storage system.} \\
\\
                    & {\bf U.S. Patent 8,010,738} & 8/2011 \\
                    & \multicolumn{2}{p{4.5in}}{\em Techniques for obtaining a specified lifetime for a data storage device} \\
                    & \multicolumn{2}{p{4.5in}}{This patent describes a method to ensure that a NAND-flash based solid state disk will be
						usable for a specified period of time.} \\
\\
                    & {\bf U.S. Patent 8,375,187} & 2/2013 \\
                    & \multicolumn{2}{p{4.5in}}{\em I/O scheduling for flash drives} \\
                    & \multicolumn{2}{p{4.5in}}{This patent describes a technique called {\em modal writes} that boosts the performance of
		    a RAID group made up of flash drives.} \\
\\
                    & {\bf U.S. Patent 8,566,553} & 10/2013 \\
                    & {\bf U.S. Patent 8,966,216} & 2/2015 \\
                    & \multicolumn{2}{p{4.5in}}{\em Techniques for automated evaluation and movement of data between storage tiers} \\
                    & \multicolumn{2}{p{4.5in}}{These patents describe a technique for managing data movement in a tiered storage array.
		    EMC Corp.'s VMAX storage array, industry's first tiered storage array, sold software based on this technique under
		    the brand name {\em FAST v1}.} \\
\\
                    & {\bf U.S. Patent 8,868,798} & 10/2014 \\
                    & \multicolumn{2}{p{4.5in}}{\em Techniques for modeling disk performance} \\
                    & \multicolumn{2}{p{4.5in}}{This patent describes a technique for modeling disk drive performance in a tiered storage
		    system. The technique was used in EMC VMAX tiered storage array software sold under the brand name {\em FAST v2}.} \\
\\
\newpage
                    & {\bf U.S. Patent 8,838,887} & 9/2014 \\
                    & \multicolumn{2}{p{4.5in}}{\em Drive partitioning for automated storage tiering} \\
                    & {\bf U.S. Patent 8,976,636} & 3/2015 \\
                    & \multicolumn{2}{p{4.5in}}{\em Techniques for storing data on disk drives partitioned into two regions} \\
                    & \multicolumn{2}{p{4.5in}}{These patents describe a technique for storing data on a hard disk drive in such a way that
		    improves both average response time as well as the throughput.} \\
\\

{\bf Publications}  & \multicolumn{2}{p{4.5in}}{\bf Passion: Optimized I/O for Parallel Applications} \\
                    & \multicolumn{2}{p{4.5in}}{\em IEEE Computer, June 1996}\\
                    & \multicolumn{2}{p{4.5in}}{Sachin More, Rajeev Thakur, Alok Choudhary, Rajesh Bordawekar, Sivaramakrishna Kuditipudi}\\
\\
                    & \multicolumn{2}{p{4.5in}}{\bf MTIO - A Multi-Threaded Parallel I/O System}\\
                    & \multicolumn{2}{p{4.5in}}{\em 11th International Parallel Processing Symposium, April 1997}\\
                    & \multicolumn{2}{p{4.5in}}{Sachin More, Alok Choudhary, Ian Foster, Ming Xu} \\
\\
                    & \multicolumn{2}{p{4.5in}}{\bf Efficient Sequencing Tape-Resident Jobs}\\
                    & \multicolumn{2}{p{4.5in}}{\em Proceedings of the Eighteenth ACM SIGACT-SIGMOD-SIGART Symposium on Principles of Database Systems, June 1999}\\
                    & \multicolumn{2}{p{4.5in}}{Sachin More, S. Muthukrishnan, Elizabeth Shriver}\\
\\
                    & \multicolumn{2}{p{4.5in}}{\bf Data Management for Large-Scale Scientific Computations in High Performance Distributed Systems}\\
                    & \multicolumn{2}{p{4.5in}}{\em The Eighth IEEE International Symposium on High Performance Distributed Computing, August 1999}\\
                    & \multicolumn{2}{p{4.5in}}{Sachin More, Alok Choudhary, Mahmut T. Kandemir, Harsha S. Nagesh, Jaechun No, Xiaohui Shen, Valerie E. Taylor, Rajeev Thakur}\\
\\
                    & \multicolumn{2}{p{4.5in}}{\bf Data Management for Large-Scale Scientific Computations in High Performance Distributed Systems}\\
                    & \multicolumn{2}{p{4.5in}}{\em Cluster Computing, Volume 3, November 2000}\\
                    & \multicolumn{2}{p{4.5in}}{Sachin More, Alok Choudhary, Mahmut T. Kandemir, Jaechun No, Gokhan Memik, Xiaohui Shen,
						 Wei-keng Liao, Harsha S. Nagesh, Valerie E. Taylor, Rajeev Thakur, Rick L. Stevens} \\
\\
                    & \multicolumn{2}{p{4.5in}}{\bf Tertiary Storage Organization for Large Multidimensional Datasets}\\
                    & \multicolumn{2}{p{4.5in}}{\em Seventeenth IEEE Symposium on Mass Storage Systems, March 2000}\\
                    & \multicolumn{2}{p{4.5in}}{Sachin More, Alok Choudhary}\\
\\
                    & \multicolumn{2}{p{4.5in}}{\bf A novel application development environment for large-scale scientific computations}\\
                    & \multicolumn{2}{p{4.5in}}{\em Proceedings of the 2000 International Conference on Supercomputing, May 2000}\\
                    & \multicolumn{2}{p{4.5in}}{Sachin More, Xiaohui Shen, Wei-keng Liao, Alok Choudhary, Gokhan Memik, Mahmut T. Kandemir, George K. Thiruvathukal, Arti Singh}\\
\\
                    & \multicolumn{2}{p{4.5in}}{\bf Scheduling Queries for Tape-Resident Data}\\
                    & \multicolumn{2}{p{4.5in}}{\em 6th International Euro-Par Conference, August 2000}\\
                    & \multicolumn{2}{p{4.5in}}{Sachin More, Alok Choudhary}\\
\end{longtable}
\end{document}
