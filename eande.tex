\section{Work Experience}

\cventry{2/2016 - 3/2020}{Consultant Technologist}{Infinidat Inc.}{Waltham, MA}{Platform Research and Development}
{
	\begin{itemize}
		\item Defined and implemented hardware qualification processes for the
			off-the-shelf hardware components used to manufacture the flagship
			product. Hardware included x86-based Servers, Fibre channel HBAs,
			SAS HBAs, HDDs, SSDs etc.  The development was done in \texttt{C/C++/Python}
			on a \texttt{CentOS Linux} platform. The code was maintained under \texttt{GitLab}.
		\item Participated in internal design and code reviews; engaged in
			recurring calls with selected vendors. Followed up specific issues 
			with vendors, including but not limited to, firmware bugs and
			suspected design issues.
		\item  Monitored the industry for new technologies.
		\item Advised the manufacturing division on Integration and Verification Testing (IVT).
	\end{itemize}
}
\cventry{4/2013 - 1/2016}{Consultant Technologist}{EMC Corporation}{Hopkinton, MA}{}
{
	\begin{itemize}
		\item Represented the EMC VMAX Division as subject matter expert for HDD/SSD to vendors
			and the rest of the organization.
		\item Invented an algorithm for scheduling I/Os to SSDs to improve
			their performance in a RAID configuration.
		\item Designed a new data layout scheme for hard disk drives to improve
			their performance in a tiered storage system.
			The scheme was implemented and tested using an I/O driver written
			in \texttt{C/C++} on \texttt{CentOS Linux} platform using the \texttt{sg} device driver.
	\end{itemize}
}
\cventry{3/2012 - 4/2013}{Consulting Performance Engineer}{EMC Corporation}{Hopkinton, MA}
{Innovation and Systems Engineering, Enterprise Storage Division}
{
	\begin{itemize}
        \item Proposed and implemented the {\em Disk Cloning Appliance} used for bulk
            replacement of faulty disks in a storage array. The appliance was
			developed in \texttt{C++} under \texttt{CentOS Linux} and used \texttt{GTK+} for the user
			interface.
		\item Implemented a simulator to evaluate the proposed metadata paging
			scheme in EMC VMAX Storage System. The software was written in
			\texttt{C++} on a \texttt{CentOS Linux} platform.
	\end{itemize}
}
\cventry{12/2006 - 3/2012}{Principal Performance Engineer}{EMC Corporation}{Hopkinton, MA}{Innovation and Systems
Engineering, Enterprise Storage Division}
{ 
	\begin{itemize}
		\item Designed and implemented the {\em Secure Erase Appliance for SSDs}
			for the {\em EMC certified secure erase service} as part of the
			launch of SSDs in the EMC storage products lineup. The appliance was
			developed in \texttt{C++} under \texttt{CentOS Linux} and used 
			\texttt{GTK+} for the user interface.
		\item Member of the team that developed tiered storage software for
			EMC VMAX marketed under the brand name \texttt{FAST v1}.
		\item Defined the disk performance modeling framework for tiered storage
			software for EMC VMAX marketed under the brand name \texttt{FAST v2}.
	\end{itemize}
}
\cventry{12/2003 - 12/2006}{Principal Performance Engineer}{EMC Corporation}{Hopkinton, MA}{Performance Research Group,
Symmetrix Business Unit}
{
	\begin{itemize}
		\item Was member of the team that designed the new cache management
			subsystem in the EMC Symmetrix Storage System.
		\item Proposed, prototyped and evaluated modifications to the RAID data rebuild algorithm
			in the EMC Symmetrix Storage System to improve its performance when
			the storage system is under heavy load. The
			development was done in \texttt{C} under a proprietary BSD-derived RTOS.
		\item Maintained Cache Simulator software that simulated the behavior of the 
			shared memory cache model used by the EMC Symmetrix/VMAX storage system.
			The software was originally written in \texttt{C}. Ported it to \texttt{C++}.
		\item Developed and maintained Hard Disk Simulator software that
			simulated the behavior of different hard disk drives used by the
			EMC Symmetrix/VMAX storage system. The software was written in \texttt{C++}.
	\end{itemize}
}
\cventry{8/2000 - 12/2003}{Senior Performance Engineer}{EMC Corporation}{Hopkinton, MA}{Performance Research Group,
Symmetrix Business Unit}
{
	\begin{itemize}
		\item Proposed, prototyped and evaluated changes to how the EMC Symmetrix Storage System manages dirty data.
			The development was done in \texttt{C} under a proprietary BSD-derived RTOS.
		\item Proposed, prototyped and evaluated a new algorithm in the EMC Symmetrix Storage System schedules read
			operations on disk drives in \texttt{RAID-1} configuration.
			The development was done in \texttt{C} under a proprietary BSD-derived RTOS.
		\item Proposed, prototyped and evaluated enhancement to the existing data pre-fetching algorithm
			in the EMC Symmetrix Storage System for the cases where workloads contain short bursts of 
			contiguous data accesses. The development was done in \texttt{C} under a proprietary BSD-derived RTOS.
		\item Prototyped and evaluated changes to the multi-LRU caching mechanism in the EMC Symmetrix Storage
			System to strike a better balance between the performance (retaining frequently accessed data) and the cost
			(memory accesses needed to retain data in the cache).
			The development was done in \texttt{C} under a proprietary BSD-derived RTOS.
	\end{itemize}
}
\cventry{9/1996 - 6/2000}{Research Assistant}{Department of Electrical and Computer Engineering}
{Northwestern University}{Evanston, IL}
{\begin{itemize}
	\item System administrator for an IBM SP2 and SGI Origin belonging to the 
		Center for Parallel and Distributed Computing at Northwestern University.
	\item Researched methods for efficient access to large objects in an object-oriented
		database running on a distributed memory parallel computer. The
		software was written in \texttt{C++}.
	\end{itemize}}
\cventry{10/1994 - 8/1996}{Research Assistant}{Department of Electrical and
Computer Engineering}{Syracuse University}{Syracuse, NY}
{Developed and evaluated a multi-threaded run-time library for parallel I/O that
separated the compute and I/O tasks in two separate threads of control. The
software was written in \texttt{C}.}
\cventry{2/1994 - 7/1994}{Software Engineer}{Datamatics Ltd.}{Mumbai, India}{}
{Worked on a document sharing system over WAN. The software was written in \texttt{C}.}
\cventry{2/1993 - 1/1994}{System Executive}{Spectrum Business Support Pvt. Ltd.}
{Mumbai, India}{}{Was member of a team that implemented a archive retrieval system.
Wrote the communication layer of the software that connected the server running
on a SUN system to the client PC over a telephone line. The software was
written in \texttt{C}.}

\section{Education}   
\cventry{1996--2000}{Doctor of Philosophy}{Northwestern University}{Evanston, IL}{}{Computer Engineering}
\cventry{1994--1996}{Master of Science}{Syracuse University}{Syracuse, NY}{}{Computer Engineering}
\cventry{1987--1991}{Bachelor of Engineering}{Savitribai Phule Pune University}{Pune, India}{}{Computer Engineering}
\section{Patents and Publications}
\cventry{}{Awarded more than 30 patents for inventions in the enterprise data storage field}{}{}{}{}
\cventry{}{Published 8 peer-reviewed research papers related to data storage}{}{}{}{}
\section{Awards and Recognition}
\cventry{2008}{President's Award at EMC Corp.}{For innovations related to solid state drive technology.}{}{}{}
